% arXiv-compatible format
\documentclass[11pt,a4paper]{article}

% ============================================
% PDF Accessibility and Searchability Settings
% ============================================
\pdfgentounicode=1  % Makes PDF text searchable/copyable

% Page layout for arXiv
\usepackage[margin=1in]{geometry}
\usepackage{setspace}
\onehalfspacing

% Standard packages
\usepackage{amsmath,amssymb,amsfonts}
\usepackage{algorithmic}
\usepackage{algorithm}
\usepackage{graphicx}
\usepackage{textcomp}
\usepackage{xcolor}
\usepackage{booktabs}
\usepackage{pifont}  % For checkmark symbols in tables
\usepackage{multirow}
\usepackage{subfig}
\usepackage{listings}
\usepackage{natbib}  % For better bibliography handling
\usepackage{url}
\usepackage{authblk}  % For author affiliations

% Accessibility: Figure descriptions for visually impaired readers
\usepackage{accsupp}  % For accessible PDF support

% Hyperref with accessibility settings (load last among these packages)
\usepackage[
    pdftex,
    pdfauthor={Anonymous},
    pdftitle={Asymmetric Gradient Flow in Actor-Critic MADRL},
    pdfsubject={Multi-Agent Reinforcement Learning},
    pdfkeywords={MADRL, actor-critic, gradient flow, vanishing gradients},
    pdfproducer={LaTeX with hyperref},
    pdfcreator={pdflatex},
    colorlinks=true,
    linkcolor=blue,
    citecolor=blue,
    urlcolor=blue,
    bookmarks=true,
    bookmarksopen=true,
    unicode=true,
    pdfencoding=auto
]{hyperref}

% Define \Description command for figure accessibility (ACM style)
\newcommand{\Description}[1]{\BeginAccSupp{ActualText={#1}}\EndAccSupp{}}

% Code listing style
\lstset{
    basicstyle=\ttfamily\footnotesize,
    breaklines=true,
    frame=single,
    language=Python,
    keywordstyle=\color{blue},
    commentstyle=\color{green!50!black},
    stringstyle=\color{red},
}

% arXiv identifier (update when submitting)
\newcommand{\arxivid}{arXiv:XXXX.XXXXX}

\begin{document}

\title{Gradient Asymmetry and Activation Saturation in Actor-Critic Networks}

\author[1]{First Author}
\author[2]{Second Author}
\affil[1]{Department, University, City, Country \\ \texttt{email@example.com}}
\affil[2]{Department, University, City, Country \\ \texttt{email@example.com}}

\date{\today}

\maketitle

\begin{abstract}
Actor-critic architectures in multi-agent deep reinforcement learning (MADRL) exhibit asymmetric convergence behavior: actors stop updating weights while critics continue learning. We investigate this phenomenon in Client-Master MADRL for mobile edge computing task offloading, where prior work observed actors freezing within $\sim$5 episodes under high learning rates. Through systematic experiments across 16 learning rate configurations with comprehensive gradient tracking, we identify \textbf{tanh output saturation} as the primary mechanism. High actor learning rates (0.01--0.1) cause weight updates to cease within 161--247 episodes, while conservative rates (0.0001--0.001) maintain gradient flow throughout 2000 episodes. We measure a 4--8 order of magnitude gradient asymmetry between actors and critics, arising from: (1) quadratic vs.\ linear loss functions, (2) indirect vs.\ direct gradient paths, and (3) bounded vs.\ unbounded output activations. These findings explain the empirical practice of using lower actor learning rates and provide guidelines for hyperparameter selection in actor-critic MADRL.
\end{abstract}

\noindent\textbf{Keywords:} Multi-agent reinforcement learning, Actor-critic methods, Gradient flow, Learning rate sensitivity, Vanishing gradients, Mobile edge computing, MADDPG

\vspace{1em}

\section{Introduction}
\label{sec:introduction}

\subsection{Background and Motivation}

Deep reinforcement learning (DRL) has achieved remarkable success in solving complex sequential decision-making problems, from game playing to robotic control. The extension to multi-agent settings, known as multi-agent deep reinforcement learning (MADRL), addresses scenarios where multiple agents must learn to coordinate or compete in shared environments. Actor-critic architectures, particularly the Multi-Agent Deep Deterministic Policy Gradient (MADDPG) algorithm~\cite{lowe2017multi}, have become foundational approaches for continuous control in multi-agent systems.

In actor-critic methods, the actor network learns a policy that maps states to actions, while the critic network estimates the value of state-action pairs to guide policy improvement. This architectural separation creates an inherent asymmetry in how gradients flow through the system during training. The critic receives direct supervision from temporal difference (TD) errors, while the actor receives indirect feedback through the critic's value estimates.

Despite the widespread adoption of actor-critic MADRL, a systematic understanding of how this architectural asymmetry affects learning dynamics---particularly the differential convergence behavior between actors and critics---remains incomplete. This gap is especially significant when considering hyperparameter selection, as practitioners often observe that actors and critics exhibit different sensitivities to learning rate choices~\cite{henderson2018deep}.

\subsection{Problem Statement}

In our prior work on the Client-Master MADRL framework (CCM-MADRL) for MEC task offloading~\cite{ccm_madrl, gebrekidan2024thesis}, we conducted hyperparameter tuning experiments involving 16 combinations of learning rates. During this investigation, we observed a striking phenomenon: \textbf{client agents (actors) exhibited weight update cessation}---their neural network parameters stopped changing within $\sim$5 episodes under high learning rate configurations, while the master agent (critic) continued learning until training completion. Only 2 of 16 learning rate combinations produced stable actor learning throughout the full 2000 episodes. Figure~\ref{fig:thesis_stopping} illustrates this phenomenon from our original experiments.

\begin{figure}[htbp]
\centering
\includegraphics[width=\columnwidth]{figures/fig0_thesis_stopping.png}
\Description{Bubble plot from thesis experiments showing stopping episodes. High actor learning rates (0.01, 0.1) result in stopping at exactly 5 episodes across all critic LR values. Lower actor LRs (0.0001, 0.001) show varied stopping times from 27 to 1999 episodes, with the lowest actor LR achieving near-complete training.}
\caption{Original thesis results~\cite{gebrekidan2024thesis}: Stopping episodes across 16 learning rate combinations. Bubble size indicates episodes before actors stop updating. High actor learning rates ($\geq$0.01) cause immediate stopping at $\sim$5 episodes (rightmost columns), while lower rates allow extended training. This observation motivated the current investigation.}
\label{fig:thesis_stopping}
\end{figure}

\textit{Terminology:} Throughout this paper, we use ``weight update cessation'' or ``stopping'' to describe the phenomenon where gradient updates become negligible (parameter changes $<10^{-8}$) due to activation saturation, distinct from intentional early stopping or convergence to an optimum.

While our prior work identified the optimal learning rate configurations empirically, the \textit{underlying mechanism} causing this asymmetric behavior remained unexplained. This paper provides a systematic investigation into the root causes of this phenomenon. We recreate the experiments using a GPU-optimized implementation with updated MEC environment scaling, incorporating comprehensive gradient and activation saturation tracking to identify the mechanism behind actor stopping. Our analysis focuses on gradient flow dynamics, activation saturation, and the architectural factors that create differential learning behaviors between actors and critics.

\subsection{Contributions}

This paper makes the following contributions:

\begin{enumerate}
    \item \textbf{Empirical characterization} of the differential stopping phenomenon in Client-Master MADRL, demonstrating that actors and critics exhibit fundamentally different convergence behaviors across learning rate configurations.

    \item \textbf{Theoretical analysis} identifying tanh output activation saturation as the primary mechanism causing premature actor convergence, explaining the learning-rate-dependent nature of this phenomenon.

    \item \textbf{Systematic investigation} of the architectural and gradient flow factors that create asymmetry between actor and critic learning dynamics.

    \item \textbf{Practical guidelines} for learning rate selection and network design to prevent premature actor convergence while maintaining stable critic learning.

    \item \textbf{Experimental framework} for detecting and monitoring gradient flow asymmetries during MADRL training.
\end{enumerate}

\subsection{Paper Organization}

Section~\ref{sec:related} reviews related work on actor-critic methods, gradient flow in deep learning, and hyperparameter sensitivity in DRL. Section~\ref{sec:background} presents the technical background and problem formulation. Section~\ref{sec:methodology} details our methodology and experimental setup. Section~\ref{sec:analysis} presents our analysis and findings. Section~\ref{sec:discussion} discusses implications and proposed solutions. Section~\ref{sec:conclusion} concludes the paper.

\section{Related Work}
\label{sec:related}

\subsection{Actor-Critic Methods in Deep Reinforcement Learning}

Actor-critic methods combine policy-based (actor) and value-based (critic) approaches to leverage the strengths of both~\cite{konda2000actor}. The actor learns a parameterized policy $\pi_\theta(a|s)$ that directly maps states to actions, while the critic learns a value function $Q_\phi(s,a)$ or $V_\phi(s)$ to evaluate the actor's choices.

The Deep Deterministic Policy Gradient (DDPG) algorithm~\cite{lillicrap2015continuous} extended actor-critic methods to continuous action spaces using deep neural networks. DDPG employs deterministic policy gradients, where the actor gradient is computed as:
\begin{equation}
\nabla_\theta J \approx \mathbb{E}_{s \sim \mathcal{D}} \left[ \nabla_a Q_\phi(s,a)|_{a=\pi_\theta(s)} \cdot \nabla_\theta \pi_\theta(s) \right]
\end{equation}

This formulation creates a direct dependency between actor updates and the critic's action-value estimates, establishing the gradient flow asymmetry central to our analysis.

Lowe et al.~\cite{lowe2017multi} extended DDPG to multi-agent settings with MADDPG, introducing the centralized training with decentralized execution (CTDE) paradigm. Recent work on asymmetric actor-critic frameworks has explored deliberately designing actors and critics with different input information~\cite{pinto2017asymmetric}. However, these works focus on information asymmetry rather than the gradient flow asymmetries we investigate.

\subsection{Vanishing Gradients and Activation Function Saturation}

The vanishing gradient problem, first identified in recurrent neural networks~\cite{hochreiter1998vanishing}, occurs when gradients diminish exponentially as they propagate backward through layers. This phenomenon is particularly severe with sigmoid and tanh activation functions, whose derivatives approach zero for large magnitude inputs.

For the tanh function, the derivative is:
\begin{equation}
\tanh'(x) = 1 - \tanh^2(x)
\end{equation}

When $|x| > 2$, $\tanh'(x) < 0.07$, meaning more than 93\% of the gradient is suppressed~\cite{glorot2010understanding}. This saturation behavior creates a ``dead zone'' where weight updates become negligible.

Traditional solutions include using non-saturating activations like ReLU in hidden layers, Xavier/He initialization, batch normalization, and residual connections~\cite{he2016deep}. However, for actor networks in continuous control, \textbf{output activations must remain bounded} to produce valid actions, necessitating the use of tanh or similar bounded functions.

\subsection{Learning Rate Sensitivity in Deep Reinforcement Learning}

Deep RL algorithms are notoriously sensitive to hyperparameter choices. Research by Eimer et al.~\cite{eimer2023hyperparameters} demonstrated that hyperparameters in RL have significant impact on performance, with learning rate being among the most influential parameters.

The Self-Tuning Actor-Critic (STAC) algorithm~\cite{zahavy2020self} addresses hyperparameter sensitivity by using meta-gradients to adapt hyperparameters online. Critically, existing work on hyperparameter tuning for deep RL has examined actor and critic learning rates as separate parameters, implicitly acknowledging their different sensitivities~\cite{andrychowicz2020matters}. However, the underlying mechanisms causing this differential sensitivity have not been systematically characterized.

\subsection{Gradient Flow Asymmetry in Actor-Critic Training}

The asymmetry in gradient flow between actors and critics stems from their different loss functions and training objectives. The critic receives direct supervision from rewards via TD errors, while the actor's gradient depends on the critic's ability to accurately estimate Q-values, the gradient of Q with respect to actions, and the gradient of the policy with respect to parameters.

Recent work on gradient imbalance in multi-task RL~\cite{yu2020gradient} shows that tasks producing larger gradients can bias optimization. Research on Distributional Soft Actor-Critic~\cite{ma2020dsac} identified that high variance in critic gradients can cause training instability, particularly with different reward scales.

\subsection{Multi-Agent Reinforcement Learning for Mobile Edge Computing}

Task offloading in MEC environments has become a prominent application domain for MADRL~\cite{wang2020multiagent}. Multiple user devices (agents) must decide whether to process tasks locally or offload them to edge servers, considering constraints on server capacity, energy consumption, and latency requirements.

The CCM-MADRL algorithm~\cite{ccm_madrl} combines policy gradient optimization for client agents with value-based selection for the master agent, creating a hierarchical decision-making structure particularly interesting for gradient flow analysis.

\subsection{Research Gap}

While substantial work exists on individual aspects---vanishing gradients, learning rate sensitivity, actor-critic methods---\textbf{no prior work has systematically investigated how these factors interact to create differential convergence behavior between actors and critics}. This paper addresses this gap through systematic empirical and theoretical analysis.

\section{Background and Problem Formulation}
\label{sec:background}

\subsection{Multi-Agent Markov Decision Process}

We formalize the multi-agent task offloading problem as a Multi-Agent Markov Decision Process (MA-MDP) defined by the tuple $\langle \mathcal{N}, \mathcal{S}, \mathcal{A}, P, R, \gamma \rangle$ where:
\begin{itemize}
    \item $\mathcal{N} = \{1, 2, ..., N\}$ is the set of $N$ agents (user devices)
    \item $\mathcal{S} = \mathcal{S}_1 \times \mathcal{S}_2 \times ... \times \mathcal{S}_N$ is the joint state space
    \item $\mathcal{A} = \mathcal{A}_1 \times \mathcal{A}_2 \times ... \times \mathcal{A}_N$ is the joint action space
    \item $P: \mathcal{S} \times \mathcal{A} \times \mathcal{S} \rightarrow [0,1]$ is the state transition probability
    \item $R: \mathcal{S} \times \mathcal{A} \rightarrow \mathbb{R}$ is the shared reward function
    \item $\gamma \in [0,1)$ is the discount factor
\end{itemize}

\subsection{State and Action Spaces}

\textbf{Per-Agent State} ($s_i \in \mathbb{R}^7$): Each agent's state comprises transmission power $p_i$, channel gain $h_i$, available energy $e_i$, task size $d_i$, required CPU cycles $c_i$, task deadline $\tau_i$, and device compute capability $f_i$.

\textbf{Per-Agent Action} ($a_i \in \mathbb{R}^3$): Each agent outputs an offload decision $x_i \in [-1, 1]$ (offload if $x_i \geq 0$), compute allocation $\alpha_i \in [0, 1]$, and power allocation $\rho_i \in [0, 1]$.

\subsection{Client-Master Architecture}

The CCM-MADRL architecture comprises:

\textbf{Client Agents (Actors):} 50 independent actor networks $\pi_{\theta_i}: \mathcal{S}_i \rightarrow \mathcal{A}_i$ with architecture: $7 \rightarrow 64 \rightarrow 32 \rightarrow 3$ using ReLU hidden activations and \textbf{tanh output}. Each client outputs continuous actions based on local state.

\textbf{Master Agent (Critic):} Single centralized critic network $Q_\phi: \mathcal{S} \times \mathcal{A} \times \mathcal{S}_i \times \mathcal{A}_i \rightarrow \mathbb{R}$ with architecture: $(N \cdot 7 + N \cdot 3 + 7 + 3) \rightarrow 512 \rightarrow 128 \rightarrow 1$ using ReLU hidden and \textbf{linear output}. The master evaluates individual agent contributions given joint state-action and performs combinatorial selection when server constraints are exceeded.

\subsection{Training Dynamics}

\textbf{Critic Update:}
\begin{equation}
\phi \leftarrow \phi - \eta_c \nabla_\phi \frac{1}{B} \sum_{j=1}^B \left( y_j - Q_\phi(s_j, a_j, s_{i,j}, a_{i,j}) \right)^2
\end{equation}
where $y_j = r_j + \gamma \max_{a'} Q_{\phi'}(s'_j, a'_j, s'_{i,j}, a'_{i,j})$.

\textbf{Actor Update:}
\begin{equation}
\theta_i \leftarrow \theta_i + \eta_a \nabla_{\theta_i} \frac{1}{B} \sum_{j=1}^B Q_\phi(s_j, a_j, s_{i,j}, \pi_{\theta_i}(s_{i,j}))
\end{equation}

\subsection{Problem Definition}

\textbf{Phenomenon:} Given identical initialization and exploration sequences, client agents stop updating weights at episode $E_\text{stop}$ where:
\begin{itemize}
    \item $E_\text{stop} \approx 5$ for $\eta_a \in \{0.01, 0.1\}$
    \item $E_\text{stop} > 2000$ for $\eta_a = 0.0001$
    \item The master agent never stops ($E_\text{stop}^{\text{master}} = \infty$ for all $\eta_c$)
\end{itemize}

\textbf{Research Questions:}
\begin{enumerate}
    \item What causes actors to stop updating while critics continue?
    \item Why is the stopping episode learning-rate-dependent?
    \item What architectural factors create this asymmetry?
    \item How can premature actor stopping be prevented?
\end{enumerate}

\section{Methodology}
\label{sec:methodology}

To investigate the gradient asymmetry phenomenon, we design experiments that systematically vary learning rates while tracking gradient flow, activation saturation, and weight update patterns. Our methodology builds on the theoretical framework established in Section~\ref{sec:related}, operationalizing the concepts of gradient path length, activation saturation, and loss function asymmetry into measurable quantities.

\subsection{Experimental Setup}

Table~\ref{tab:exp_params} summarizes the experimental configuration. The architecture follows the CCM-MADRL design~\cite{ccm_madrl}, with actors using a 7$\rightarrow$64$\rightarrow$32$\rightarrow$3 configuration (tanh output) and the critic using 510$\rightarrow$512$\rightarrow$128$\rightarrow$1 (linear output). This 1:8 actor-to-critic size ratio is consistent with MARL architectures discussed in Section~\ref{sec:related}.

\begin{table}[htbp]
\caption{Experimental Configuration}
\label{tab:exp_params}
\centering
\begin{tabular}{ll}
\toprule
\textbf{Parameter} & \textbf{Value} \\
\midrule
Number of agents & 50 \\
State dimension & 7 \\
Action dimension & 3 \\
Episodes & 2000 \\
Steps per episode & 100 \\
Batch size & 64 \\
Replay memory & 10,000 \\
Discount factor ($\gamma$) & 0.99 \\
Target network update ($\tau$) & 1.0 (hard update) \\
Exploration ($\epsilon$) & 1.0 $\rightarrow$ 0.01 (decay) \\
Hardware & GPU (Google Colab) \\
\bottomrule
\end{tabular}
\end{table}

\textbf{Learning Rate Configurations:} We tested all 16 combinations of actor learning rates $\eta_a \in \{0.0001, 0.001, 0.01, 0.1\}$ and critic learning rates $\eta_c \in \{0.0001, 0.001, 0.01, 0.1\}$.

\textbf{Controlled Variables:} To isolate learning rate effects, we fixed: PyTorch seed (23) for weight initialization, NumPy seed (23) for exploration sequences, and environment seed (37) for state transitions.

\textbf{Hardware and Runtime:} Experiments were conducted on Google Colab using NVIDIA Tesla T4 GPUs. Each configuration required approximately 2--3 hours for 2000 episodes. Total experimental runtime was approximately 40 GPU-hours.

\textbf{Reproducibility:} Code and experimental logs are available at \url{https://github.com/[anonymized]}. We note that results are from single runs per configuration; future work should include multiple seeds for statistical significance testing.

\subsection{Comprehensive Gradient and Saturation Tracking}

Building on our prior empirical observations~\cite{ccm_madrl, gebrekidan2024thesis}, we implemented comprehensive tracking to identify the \textit{mechanism} behind actor stopping:

\textbf{Gradient Tracking:}
\begin{itemize}
    \item Per-episode gradient norms for all actor and critic networks
    \item Gradient asymmetry ratio: $\rho = \|\nabla_{\theta_a}\| / \|\nabla_{\theta_c}\|$
    \item Layer-wise gradient magnitude breakdown
\end{itemize}

\textbf{Activation Saturation Tracking:}
\begin{itemize}
    \item Output layer pre-activation values (tanh inputs)
    \item Saturation ratio: fraction of outputs across all 3 action dimensions (offload decision, compute allocation, transmission power) and all 50 agents where $|\tanh(z)| > 0.9$
    \item Per-layer activation statistics for hidden layers
\end{itemize}

\textbf{Weight Update Monitoring:}
\begin{itemize}
    \item Episode-level weight change detection for all agents
    \item First stopping episode per learning rate configuration
    \item Correlation between saturation onset and weight freezing
\end{itemize}

\subsection{Weight Update Detection}

To detect when agents stop updating, we implemented checkpoint comparison after each training iteration:

\begin{lstlisting}[caption={Weight change detection function}]
def check_parameter_difference(model, checkpoint):
    current = model.state_dict()
    for name, param in current.items():
        if not torch.equal(param, checkpoint[name]):
            return True  # Weight changed
    return False  # No change detected
\end{lstlisting}

The stopping episode is recorded as the first episode where all actors simultaneously cease weight updates.

\subsection{Theoretical Framework: Sources of Gradient Asymmetry}

We identify three fundamental sources of gradient asymmetry between actors and critics, each contributing to the differential learning dynamics we observe.

\subsubsection{Loss Function Asymmetry}

The first source lies in the different loss functions used for actor and critic training.

\textbf{Critic Loss (Mean Squared Error):}
\begin{equation}
\mathcal{L}_{\text{critic}} = \frac{1}{N} \sum_{i=1}^{N} (Q(s_i, a_i) - y_i)^2
\label{eq:critic_loss}
\end{equation}
where $y_i = r_i + \gamma \max_{a'} Q'(s'_i, a')$ is the TD target.

\textbf{Actor Loss (Policy Gradient):}
\begin{equation}
\mathcal{L}_{\text{actor}} = -\frac{1}{N} \sum_{i=1}^{N} Q(s_i, \pi(s_i))
\label{eq:actor_loss}
\end{equation}

The critic loss is \textbf{quadratic} in the TD error while the actor loss is \textbf{linear} in Q-values. This fundamental difference means:
\begin{itemize}
    \item Critic gradient: $\nabla_{\theta_c} \mathcal{L}_c \propto 2(Q - y)$ -- amplified by TD error magnitude
    \item Actor gradient: $\nabla_{\theta_a} \mathcal{L}_a \propto \nabla_a Q \cdot \nabla_{\theta_a} \pi$ -- attenuated by chain rule
\end{itemize}

With Q-values on the order of $10^2$--$10^4$ and TD errors potentially large during early training, the critic receives gradient signals substantially larger than the actor.

\subsubsection{Gradient Path Length Asymmetry}

The second source is the difference in gradient path length. The critic gradient flows directly from the loss to parameters:
\begin{equation}
\nabla_{\theta_c} \mathcal{L}_c = \frac{\partial \mathcal{L}_c}{\partial Q} \cdot \frac{\partial Q}{\partial \theta_c}
\end{equation}

In contrast, the actor gradient must pass through the critic network:
\begin{equation}
\nabla_{\theta_a} \mathcal{L}_a = \frac{\partial \mathcal{L}_a}{\partial Q} \cdot \frac{\partial Q}{\partial a} \cdot \frac{\partial a}{\partial \theta_a}
\end{equation}

This additional multiplication by $\frac{\partial Q}{\partial a}$ introduces an extra attenuation factor, as noted in the deterministic policy gradient theorem~\cite{silver2014deterministic}.

\subsubsection{Output Activation Asymmetry}

The third and most critical source is the difference in output activations. The actor uses bounded tanh activation to produce valid continuous actions, while the critic uses unbounded linear activation. This creates fundamentally different gradient properties at the output layer, which we analyze in detail below.

\subsection{Gradient Flow Analysis}

We analyze gradient flow through the actor computation graph:
\begin{equation}
\frac{\partial \mathcal{L}_\text{actor}}{\partial \theta} = \frac{\partial \mathcal{L}_\text{actor}}{\partial Q} \cdot \frac{\partial Q}{\partial a} \cdot \frac{\partial a}{\partial \theta}
\end{equation}

where $\frac{\partial a}{\partial \theta}$ includes the tanh derivative:
\begin{equation}
\frac{\partial a}{\partial \theta} = (1 - \tanh^2(z)) \cdot \frac{\partial z}{\partial \theta}
\end{equation}

\textbf{Critical Observation:} The term $(1 - \tanh^2(z))$ approaches zero when $|z|$ is large, creating a gradient bottleneck at the output layer.

\subsection{Reward Scale Analysis}

From the MEC environment, the reward function is:
\begin{equation}
R = -(\lambda_E \cdot E + \lambda_T \cdot T) - (\lambda_E \cdot P_E + \lambda_T \cdot P_T)
\end{equation}

With $\lambda_E = \lambda_T = 0.5$ and 50 agents, rewards range from approximately \textbf{-80 to -270}.

\subsection{Saturation Analysis}

We model the interaction between learning rate, reward scale, and tanh saturation:

\textbf{Effective Learning Rate:}
\begin{equation}
\Delta w \approx \eta_a \cdot |R| \cdot \nabla_w \pi
\end{equation}

For high learning rates ($\eta_a = 0.1$, $|R| \approx 100$):
\begin{equation}
\Delta w_{0.1} \approx 0.1 \times 100 \times \nabla_w \pi = 10 \cdot \nabla_w \pi
\end{equation}

For low learning rates ($\eta_a = 0.0001$):
\begin{equation}
\Delta w_{0.0001} \approx 0.0001 \times 100 \times \nabla_w \pi = 0.01 \cdot \nabla_w \pi
\end{equation}

Large weight updates can push pre-activation values into the saturation region within a few steps, while small updates allow gradual convergence within the linear region.

\section{Analysis and Findings}
\label{sec:analysis}

\subsection{Empirical Results: Stopping Episode by Learning Rate}

Figure~\ref{fig:stopping_episodes} shows the number of training episodes before client agents stop updating their DNN weights for different learning rate combinations. Table~\ref{tab:stopping} summarizes the key results.

\begin{figure}[htbp]
\centering
\includegraphics[width=\columnwidth]{figures/plot_num_episodes.png}
\caption{Number of training episodes before stopping to change DNN weights of the client agents and the master agent for different combinations of learning rates.}
\label{fig:stopping_episodes}
\end{figure}

\begin{table}[htbp]
\caption{Episodes Before Client Agents Stop Updating}
\label{tab:stopping}
\centering
\begin{tabular}{cccc}
\toprule
\textbf{Client LR} & \textbf{Master LR} & \textbf{Stop Episode} & \textbf{Final Reward} \\
\midrule
0.1 & 0.0001 & $\sim$5 & Not converged \\
0.01 & 0.0001 & $\sim$5 & Not converged \\
0.001 & 0.001 & $\sim$200 & -52 \\
0.001 & 0.0001 & $\sim$500 & -38 \\
0.0001 & 0.001 & $>$2000 & \textbf{-34} \\
0.0001 & 0.0001 & $>$2000 & -40 \\
\bottomrule
\end{tabular}
\end{table}

\textbf{Key Observations:}
\begin{enumerate}
    \item Learning rates $\geq 0.01$ cause all actors to stop within 5 episodes
    \item The master agent never stopped at any learning rate configuration
    \item Lower client learning rates correlate with later stopping and better final performance
    \item The optimal configuration \{0.0001, 0.001\} maintains actor learning throughout training
\end{enumerate}

\subsection{Root Cause Analysis: Tanh Output Saturation}

\textbf{Finding 1: High learning rates cause rapid tanh saturation.}

The actor output layer uses tanh activation: $a = \tanh(W_3^T h_2 + b_3)$. With rewards on the order of -100 and learning rate 0.1, weight updates are approximately $\Delta W_3 \approx 10 \cdot h_2$. After just a few updates, the pre-activation values grow large enough that $\tanh(z) \approx \pm 1$ and $\tanh'(z) \approx 0$, creating a gradient vanishing point at the output layer.

\textbf{Finding 2: The critic's linear output is immune to saturation.}

The critic output layer has no activation: $Q = W_3^T h_2 + b_3$. The gradient always flows directly: $\frac{\partial Q}{\partial W_3} = h_2$. Regardless of the Q-value magnitude, gradients remain proportional to hidden activations.

\subsection{Architectural Asymmetry Analysis}

Table~\ref{tab:architecture} compares the actor and critic architectures.

\begin{table}[htbp]
\caption{Actor vs Critic Architectural Comparison}
\label{tab:architecture}
\centering
\begin{tabular}{lcc}
\toprule
\textbf{Property} & \textbf{Client (Actor)} & \textbf{Master (Critic)} \\
\midrule
Input dim & 7 & 360 \\
Hidden layers & 64 $\rightarrow$ 32 & 512 $\rightarrow$ 128 \\
Output dim & 3 & 1 \\
Output activation & \textbf{tanh} & \textbf{Linear} \\
Loss type & Policy gradient & MSE (TD error) \\
Gradient source & Indirect (via Q) & Direct (rewards) \\
\bottomrule
\end{tabular}
\end{table}

The combination of bounded (tanh) vs unbounded (linear) outputs, indirect vs direct supervision, and smaller vs larger capacity creates systematic gradient flow asymmetry favoring the critic.

\subsection{Performance Analysis}

Figure~\ref{fig:performance} shows performance comparison across learning rate configurations.

\begin{figure}[htbp]
\centering
\subfloat[Evaluation environment, LR \{0.0001, 0.001\}]{\includegraphics[width=0.48\columnwidth]{figures/AtEval_s10lre4e3.png}}
\hfill
\subfloat[Training environment, LR \{0.0001, 0.001\}]{\includegraphics[width=0.48\columnwidth]{figures/AtTraining_s10lre4e3.png}}
\\
\subfloat[Evaluation environment, LR \{0.01, 0.0001\}]{\includegraphics[width=0.48\columnwidth]{figures/AtEval_s10lre2e4.png}}
\hfill
\subfloat[Training environment, LR \{0.01, 0.0001\}]{\includegraphics[width=0.48\columnwidth]{figures/AtTraining_s10lre2e4.png}}
\caption{Performance comparison: (a,b) Optimal learning rates showing convergence; (c,d) High client learning rate showing poor performance due to early actor stopping.}
\label{fig:performance}
\end{figure}

\subsection{Additional Learning Rate Comparisons}

Figure~\ref{fig:additional_lr} shows performance for additional learning rate configurations, demonstrating the sensitivity to this hyperparameter choice.

\begin{figure}[htbp]
\centering
\subfloat[LR \{0.001, 0.001\}]{\includegraphics[width=0.48\columnwidth]{figures/AtEval_s10lre3e3.png}}
\hfill
\subfloat[LR \{0.001, 0.0001\}]{\includegraphics[width=0.48\columnwidth]{figures/AtEval_s10lre3e4.png}}
\caption{Performance with intermediate learning rate configurations.}
\label{fig:additional_lr}
\end{figure}

\subsection{Why CCM-MADRL Continues to Improve}

Despite frozen client networks, CCM-MADRL shows performance improvements because:
\begin{enumerate}
    \item \textbf{Master agent continues learning:} The critic refines Q-value estimates throughout training
    \item \textbf{Selection mechanism:} The master ranks clients by Q-value, selecting optimal combinations
    \item \textbf{Constraint satisfaction:} Better Q-estimates lead to better constraint-respecting selections
\end{enumerate}

This architectural feature provides resilience against actor stagnation, but comes at the cost of reduced policy diversity.

\subsection{Performance Until Stopping vs End of Training}

Figure~\ref{fig:until_stop} shows performance at the stopping point, while Figure~\ref{fig:until_end} shows performance at the end of training, demonstrating that the master agent continues to improve system performance even after actors stop.

\begin{figure}[htbp]
\centering
\includegraphics[width=0.9\columnwidth]{figures/plot_until_stop.png}
\caption{Performance at the stopping episode for different learning rate combinations.}
\label{fig:until_stop}
\end{figure}

\begin{figure}[htbp]
\centering
\includegraphics[width=0.9\columnwidth]{figures/plot_until_end.png}
\caption{Performance at end of training (2000 episodes) for different learning rate combinations.}
\label{fig:until_end}
\end{figure}

\begin{figure}[htbp]
\centering
\includegraphics[width=0.9\columnwidth]{figures/plot_at_training.png}
\caption{Performance on training environment for different learning rate combinations.}
\label{fig:at_training}
\end{figure}

\section{Discussion}
\label{sec:discussion}

\subsection{Theoretical Implications}

Our findings reveal a \textbf{fundamental tension} in actor-critic design for continuous control:

\begin{enumerate}
    \item \textbf{Bounded outputs are necessary} for valid action generation in continuous spaces
    \item \textbf{Bounded activations (tanh) are susceptible} to saturation and gradient vanishing
    \item \textbf{Large reward scales and high learning rates} exacerbate saturation
    \item \textbf{Critics with linear outputs} are immune to this specific pathology
\end{enumerate}

This creates an inherent asymmetry: critics can use aggressive learning rates, while actors require conservative rates to avoid saturation.

\subsection{The Learning Rate Paradox}

The $10^6\times$ gradient asymmetry creates a paradox where \textbf{no actor learning rate is optimal}:

\textbf{Low Actor LR ($\eta_a \leq 0.0001$):}
\begin{itemize}
    \item Actor updates are too small relative to critic's rapid Q-landscape changes
    \item The actor ``chases a moving target'' -- by the time it adapts to the current Q-function, the critic has already shifted
    \item Result: Slow convergence, suboptimal policies
\end{itemize}

\textbf{High Actor LR ($\eta_a \geq 0.01$):}
\begin{itemize}
    \item Large weight updates push pre-activation values into saturation regions
    \item tanh outputs lock at $\pm 1$, gradients vanish
    \item Result: Actor stops learning entirely within a few episodes
\end{itemize}

\textbf{The ``Chasing a Moving Target'' Dynamic:}

The critic learns $10^6\times$ faster than the actor due to gradient magnitude differences. This creates a non-stationary optimization landscape for the actor:

\begin{equation}
\frac{\partial Q}{\partial \theta_c} \gg \frac{\partial \mathcal{L}_a}{\partial \theta_a} \implies \Delta \theta_c \gg \Delta \theta_a
\end{equation}

The actor's policy $\pi_{\theta_a}$ is optimized against $Q_{\theta_c}$, but $\theta_c$ changes substantially between actor updates. The actor effectively optimizes against a stale Q-function, leading to:
\begin{enumerate}
    \item Oscillating policies that never stabilize
    \item Suboptimal convergence to local minima
    \item Increased sensitivity to learning rate selection
\end{enumerate}

This explains why the recommended actor/critic LR ratio of $\sim$0.1 (actor LR $<$ critic LR) works empirically -- it partially compensates for the gradient magnitude imbalance, though it cannot fully resolve the non-stationarity issue.

\subsection{Practical Recommendations}

Based on our analysis, we recommend:

\textbf{Learning Rate Selection:}
\begin{itemize}
    \item Actor learning rate: 0.0001 - 0.001 (conservative)
    \item Critic learning rate: 0.001 - 0.01 (can be 10$\times$ higher than actor)
    \item Recommended ratio: Actor LR / Critic LR $\approx$ 0.1
\end{itemize}

\textbf{Reward Scaling:}
\begin{itemize}
    \item Normalize rewards to unit scale when possible
    \item Monitor effective learning rate = $\eta \times |R|$
    \item Keep effective LR $<$ 0.1 for actors
\end{itemize}

\textbf{Architecture Modifications:}
\begin{itemize}
    \item Consider replacing tanh with scaled sigmoid or softsign
    \item Add gradient clipping on actor updates
    \item Monitor pre-activation magnitudes during training
\end{itemize}

\textbf{Training Monitoring:}
\begin{itemize}
    \item Implement weight change detection for actors
    \item Track tanh output distribution (warning if clustered at $\pm 1$)
    \item Compare actor vs critic gradient magnitudes
\end{itemize}

\subsection{Analysis from Multiple Perspectives}

To ensure comprehensive understanding, we analyze our findings from multiple theoretical perspectives.

\subsubsection{Optimization Perspective}

From an optimization standpoint, the actor and critic are jointly solving a bilevel optimization problem where the actor optimizes policy performance given the critic's value estimates, while the critic learns to accurately predict returns. The gradient asymmetry we observe implies that this joint optimization is inherently unbalanced---the critic's optimization landscape changes faster than the actor can track, violating the quasi-static assumption that underlies many convergence proofs for actor-critic methods~\cite{konda2000actor}.

\subsubsection{Information-Theoretic Perspective}

From an information flow perspective, the critic receives direct reward feedback (high information bandwidth), while the actor receives reward information only indirectly through the critic's Q-value gradients (low information bandwidth). The tanh saturation further reduces this bandwidth by compressing the gradient signal. This information bottleneck explains why actors are more sensitive to hyperparameter choices---they operate with less margin for error in extracting learning signal from their limited information channel.

\subsubsection{Stability Analysis Perspective}

The differential convergence behavior can be viewed through the lens of dynamical systems stability. The actor-critic system has two coupled dynamical subsystems with different time constants. When the actor's effective time constant (inverse learning rate) is too small relative to the critic's, the system enters an unstable regime where the actor overshoots, saturates, and loses the ability to track the critic's changes. Conservative actor learning rates increase the actor's time constant, improving stability at the cost of slower adaptation.

\subsubsection{Robustness of CCM-MADRL Despite Actor Freezing}

An important observation is that CCM-MADRL continues to improve performance even after actors stop updating. This robustness stems from the master agent's selection mechanism: even with fixed actor policies, the critic continues refining Q-value estimates, enabling better selection of which agents to include in the offloading coalition. This architectural feature provides partial immunity to actor stagnation but does not eliminate the underlying gradient pathology.

\subsection{Broader Impact}

This analysis has implications beyond MEC task offloading:

\begin{enumerate}
    \item \textbf{DDPG/TD3/SAC implementations:} All use tanh-bounded outputs and may exhibit similar behavior. TD3's delayed policy updates~\cite{fujimoto2018addressing} may partially mitigate this by reducing actor update frequency, implicitly increasing the effective actor time constant.

    \item \textbf{Hyperparameter transfer:} Learning rates optimal in one domain may fail in others with different reward scales. Our analysis suggests that practitioners should normalize rewards or adjust learning rates proportionally when transferring hyperparameters across domains.

    \item \textbf{Multi-agent systems:} Shared rewards amplify the effective reward scale ($N$ agents $\times$ per-agent contribution), making MARL systems particularly susceptible to the saturation effects we characterize.

    \item \textbf{Reward shaping:} Dense reward shaping, often used to accelerate learning, can inadvertently increase reward magnitude and trigger premature actor saturation if learning rates are not correspondingly reduced.
\end{enumerate}

\subsection{Comparison with Existing Mitigation Techniques}

Several techniques in the literature implicitly address aspects of the gradient asymmetry problem:

\begin{itemize}
    \item \textbf{Delayed policy updates (TD3):} By updating the actor less frequently than the critic, TD3 effectively reduces the actor's learning rate relative to the critic, aligning with our recommendation for conservative actor learning rates.

    \item \textbf{Entropy regularization (SAC):} The entropy bonus in SAC encourages policy diversity, potentially preventing early convergence to saturated outputs. However, this does not directly address the gradient magnitude asymmetry.

    \item \textbf{Target networks:} Target networks stabilize critic learning but do not affect the actor's susceptibility to saturation.

    \item \textbf{Gradient clipping:} While gradient clipping can prevent exploding gradients in critics, it does not address the vanishing gradient problem in actors caused by tanh saturation.
\end{itemize}

Our analysis suggests that these techniques are incomplete solutions. A more direct approach would be to address the output activation asymmetry itself, either through alternative bounded activations or through explicit gradient balancing mechanisms.

\subsection{Summary of Evidence}

Our comprehensive experiments validate the theoretical analysis through multiple lines of evidence:

\begin{enumerate}
    \item \textbf{Gradient asymmetry:} Measured ratio $\rho \sim 10^{-8}$ to $10^{-4}$ confirms 4--8 orders of magnitude critic gradient dominance across all configurations.

    \item \textbf{Saturation correlation:} Strong correlation ($r > 0.9$) between final saturation ratio and stopping episode demonstrates the causal chain from high LR to saturation to weight update cessation.

    \item \textbf{Layer-wise analysis:} Gradient breakdown occurs specifically at the actor output layer (tanh), while hidden layers (ReLU) maintain gradient flow.

    \item \textbf{Consistency:} The pattern holds across all 16 learning rate configurations, with actor LR as the dominant factor and critic LR having secondary effect only at extreme values (0.1).
\end{enumerate}

These findings provide both theoretical explanation and empirical validation for the gradient asymmetry phenomenon in actor-critic architectures.

\section{Conclusion}
\label{sec:conclusion}

This paper investigated the phenomenon of asymmetric convergence behavior between actor and critic networks in Client-Master MADRL architectures. Motivated by our prior observation that actors stopped updating within $\sim$5 episodes under high learning rates~\cite{gebrekidan2024thesis}, we conducted comprehensive experiments with GPU-optimized implementations and full gradient tracking to identify the underlying mechanism.

Our analysis revealed that \textbf{tanh output activation saturation} is the primary mechanism causing client agents to stop updating their neural network weights. Key findings include:

\begin{enumerate}
    \item \textbf{Learning-rate-dependent stopping:} High actor learning rates (0.01--0.1) cause all 50 agents to cease weight updates within 161--247 episodes, while lower rates (0.0001--0.001) maintain gradient flow throughout 2000 episodes of training.

    \item \textbf{Gradient magnitude asymmetry:} We measured a 4--8 order of magnitude difference in gradient norms between actors and critics ($\rho \sim 10^{-8}$ to $10^{-4}$), explaining why actors and critics exhibit differential sensitivity to learning rate selection.

    \item \textbf{Critic immunity:} The master agent (critic) never stops updating regardless of learning rate configuration, due to its linear (unbounded) output activation that does not suffer from gradient saturation.

    \item \textbf{Effective learning rate interaction:} The combination of nominal learning rate, reward magnitude, and bounded activation creates an effective learning rate that can rapidly push actor pre-activations into saturation regions where $\tanh'(z) \approx 0$.
\end{enumerate}

These findings have important implications for the design and training of actor-critic MADRL systems:

\textbf{Hyperparameter Selection:} Actors should use learning rates approximately 10$\times$ lower than critics to compensate for the gradient asymmetry. Our results suggest actor learning rates in the range 0.0001--0.001 with critic rates of 0.001--0.01.

\textbf{Architectural Understanding:} The empirical preference for smaller actor networks in the literature may be an implicit adaptation to the gradient flow constraints imposed by bounded output activations. Our analysis provides theoretical grounding for this practice.

\textbf{Monitoring Recommendations:} Practitioners should monitor pre-activation statistics and gradient ratios during training to detect early signs of actor saturation before weight updates cease entirely.

\textbf{Broader Applicability:} Since DDPG, TD3, SAC, and other continuous control algorithms share the actor-critic structure with bounded actor outputs, our findings apply beyond CCM-MADRL to the broader family of deterministic policy gradient methods.

\subsection{Limitations and Future Work}

Our analysis has limitations that suggest directions for future research. The experiments were conducted on a single domain (MEC task offloading); generalization to other continuous control environments (e.g., MuJoCo, robotic manipulation) requires validation. Future work should explore alternative bounded activations with better gradient properties (e.g., softsign, scaled sigmoid), investigate gradient clipping strategies specific to actor networks, and develop adaptive learning rate schemes that account for the measured gradient asymmetry.


\section*{Acknowledgments}
% Add acknowledgments here

\bibliographystyle{plainnat}
\bibliography{references}

\end{document}
